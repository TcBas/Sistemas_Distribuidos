\documentclass{beamer}
\usepackage[utf8]{inputenc}
\usepackage{lmodern}

% Tema y colores personalizados
\usetheme{Madrid}
\usecolortheme{whale}

% Colores personalizados
\definecolor{mycolor}{RGB}{0, 128, 128} % Color personalizado para títulos
\setbeamercolor{title}{fg=white, bg=mycolor}  
\setbeamercolor{frametitle}{fg=white, bg=mycolor}  
\setbeamercolor{block title}{bg=mycolor, fg=white} 
\setbeamercolor{block body}{bg=gray!15} 
\setbeamercolor{structure}{fg=mycolor}  
\setbeamercolor{footline}{fg=white, bg=mycolor}

% Pie de página personalizado
\setbeamertemplate{footline}{
  \leavevmode%
  \hbox{%
  \begin{beamercolorbox}[wd=.8\paperwidth,ht=2.25ex,dp=1ex,leftskip=3mm]{author in head/foot}%
    \usebeamerfont{author in head/foot}\insertshortauthor
  \end{beamercolorbox}%
  \begin{beamercolorbox}[wd=.2\paperwidth,ht=2.25ex,dp=1ex,rightskip=3mm]{date in head/foot}%
    \usebeamerfont{date in head/foot}\insertshortdate\hfill\insertframenumber/\inserttotalframenumber
  \end{beamercolorbox}}%
  \vskip0pt%
}

% Información de título
\title[Racks Go Wonky]{Racks Go Wonky}
\author{Sebastian A. Tapia T. \and Gilmar Gutierrez Flores}
\institute{Escuela Profesional de Ingeniería Estadística e Informática\\Universidad Nacional del Altiplano}
\date{\today}

\begin{document}

% Portada
\begin{frame}
    \titlepage
\end{frame}





% Diapositiva: ¿Qué significa "Racks Go Wonky"?
\section{Racks Go Wonky}
\begin{frame}{¿Qué significa "Racks Go Wonky"?}
    \begin{block}{Definición}
        La frase \textbf{"Racks Go Wonky"} indica que un grupo de servidores conectados en un rack empieza a experimentar fallos significativos, principalmente en la \textbf{conectividad de red}.
    \end{block}
    \begin{exampleblock}{Impacto}
        Esto se refiere a una \textbf{pérdida del 50\% de los paquetes}, afectando la eficiencia del tráfico de datos.
    \end{exampleblock}
\end{frame}

% Diapositiva: Pérdida de Paquetes
\section{Pérdida de Paquetes}
\begin{frame}{Pérdida de Paquetes}
    \textbf{¿Qué es la pérdida de paquetes?}
    \begin{itemize}
        \item Los datos en red se dividen en pequeñas unidades llamadas \textbf{paquetes}.
        \item La \textbf{pérdida de paquetes} ocurre cuando algunos no llegan a su destino.
        \item Esto puede causar errores o tiempos de espera prolongados en las aplicaciones.
    \end{itemize}
\end{frame}

% Diapositiva: Causas de la Pérdida de Paquetes
\section{Causas de Pérdida de Paquetes}
\begin{frame}{Causas de la Pérdida de Paquetes}
    \begin{block}{Posibles Razones}
        Algunas de las causas de la pérdida de paquetes incluyen:
    \end{block}
    \begin{itemize}
        \item Problemas de cableado en el rack.
        \item Fallos en los switches de red del rack.
        \item Configuraciones de red defectuosas.
        \item Interferencias físicas o eléctricas.
        \item Sobrecarga de tráfico en la red.
    \end{itemize}
\end{frame}

% Diapositiva: Impacto de la Pérdida del 50% de Paquetes
\section{Impacto de la Pérdida de Paquetes}
\begin{frame}{Impacto de la Pérdida del 50\% de Paquetes}
    \begin{block}{Reducción de Rendimiento}
        Perder el 50\% de los paquetes significa que la mitad de los datos no completan el ciclo de transmisión.
    \end{block}
    \begin{itemize}
        \item \textbf{Reducción de rendimiento}: Latencias o fallos en las conexiones.
        \item \textbf{Reenvío de paquetes}: Aumenta la latencia y reduce el ancho de banda.
        \item \textbf{Inestabilidad del servicio}: Servicios en el rack pueden fallar intermitentemente.
    \end{itemize}
\end{frame}

% Diapositiva: Posibles Soluciones
\section{Soluciones}
\begin{frame}{Posibles Soluciones}
    \begin{block}{Soluciones Principales}
        \begin{itemize}
            \item \textbf{Diagnóstico de red}: Monitoreo para identificar el problema.
            \item \textbf{Reparación o sustitución de hardware}: Reemplazar piezas afectadas.
            \item \textbf{Balanceo de carga}: Reducir la sobrecarga distribuyendo tráfico.
            \item \textbf{Reconfiguración}: Ajustar la configuración para evitar errores.
        \end{itemize}
    \end{block}
\end{frame}

\end{document}
